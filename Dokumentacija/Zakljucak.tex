\chapter{Zaključak i budući rad}
		
		\indent U našoj smo skupini Vidoje odlučiti razviti u potpunosti novu ideju projektnog zadatka - interaktivni socijalni kalendar po uzoru na famozni Ferko prof. Marka Čupića, skrojen da pojednostavi praćenje obveza i uskladi akademski i društveni život prosječnom Ferovcu ili Ferovki. Rad na projektu sadržavao je mnoge neočekivane izazove i zapreke, no uz discipliniranu i marljivu suradnju sve su premoštene bez prevelikih muka.
		
		\indent Prvo i možda najbitnije bilo je osmisliti originalan zadatak kako bi se njegova složenost podudarala s onim zadanim. Ideju kalendara s događajima detaljno smo razradili, ubacili dodatne idejne funkcionalnosti i opcije kako bi konkurirao i predali na odobrenje nositelja predmeta. Neposredno nakon dobivanja potvrde nastavnika da je Eventko prošao provjeru, oformili smo podjedinice za frontend, backend i dokumentaciju te počeli s raspodjelom rada na incijalnim koracima projekta. Kao zajednički oslonac koristili smo rukom crtanu skicu koja prikazuje izgled početne stranice i njezino grananje na razne funkcionalnosti.
		
		\indent Tokom izrade Eventka skrenuli smo s nekih originalnih ideja poput korištenja QR kodova za dodavanje prijatelja i Google karte ukomponirane u odabir lokacije događaja, a dodali smo neke nove poput unosa kreditne kartice pri "plaćanju" za premium račun. Bilo je slučajeva u kojima se isti posao morao raditi više puta ispočetka zbog ranijeg brzanja ili neopreznosti, ali dobro planiranje svelo je ovakve instance na minimum.
		
		\indent Zajedničko stvaranje web-aplikacije u gotovo poslovnom okruženju koristilo je kao esencijalno iskustvo svakome od članova. Rijetko kada smo morali postavljati stroge vremenske rokove, no kada jesmo bili su poštovani i sav je posao obavljen na vrijeme. Eventko ima još prostora za potencijalna buduća unapređenja koja smo mogli ostvariti uz malo više vremena i vještine, poglavito stvaranje mobilne aplikacije da ga približi idejnim korisnicima. Zadovoljni smo vremenom uloženim u projekt, konačnim rezultatom te znanjem koje smo stekli radeći ga.
		
		\eject 
\chapter{Arhitektura i dizajn sustava}
		
	\vspace{-1cm}
	\begin{figure}[h]
		\begin{center}
			\includegraphics{slike/arhitektura_skica.png}
			\caption{Arhitektura sustava}
		\end{center}	
	\end{figure}
	
	\indent Arhitekturu tvore tri podsustava: web poslužitelj, web aplikacija te baza podataka. \textit{Web preglednik} je program za pregledavanje i navigaciju web-stranicama. Kada korisnik pošalje zahtjev za web-stranicom, preglednik dohvaća potrebne datoteke s \textit{web poslužitelja} i prikazuje stranicu na korisnikovom ekranu u namijenjenom obliku. Poslužitelj omogućuje komunikaciju klijenta s \textit{web aplikacijom} koja je na njemu pokrenuta, a prosljeđuje joj zahtjeve HTTP-om (engl. \textit{Hyper Text Transfer Protocol}). Web aplikacija odgovara na zahtjeve klijenta pristupajući po potrebi bazi podataka i vraćajući HTML dokument čitljiv u web pregledniku. \\
	
	\indent Za izradu ovog projekta koristili smo se Spring Boot frameworkom u Javi kroz razvojno okruženje IntelliJ Community Edition, Javascriptom uz React u Visual Studio Code-u te nizom drugih programa za dizajn slika i grafova (GIMP, AstahUML itd.). \\
	
	\indent Arhitektura sustava prati MVC obrazac, odnosno Model-Pogled-Nadglednik (engl. \textit{Model View Controller}), stilističku varijaciju arhitekture zasnovane na događajima. Takve arhitekture odlikuje to što se komponente međusobno ne pozivaju eksplicitno, već neke od njih generiraju signale (događaje) ne znajući koja druga "osluškuje" tj. očekuje takav signal i na njega reagira. Kod MVC-a pogodno je što smanjuje međuovisnost korisničkog sučelja i ostatka sustava, a omogućuje i nezavisan razvoj, nadogradnje i dodavanje različitih dijelova aplikacije. Sadrži različite gotove predloške za klase koji nam olakšavaju proces izrade.\\
	
	\indent MVC model sastoji se od komponenti:
	\begin{packed_item}
		\item 	\textbf{Model} - Središnja komponenta sustava, sadrži razrede čiji se objekti obrađuju. Rukuje s podatkovnom logikom i bazom podataka. Prima podatke od nadglednika.
		\item 	\textbf{Pogled} - Predstavlja model korisniku na čitljiv način. Sadrži razrede čiji objekti služe za prikaz podataka. Dinamički se osvježava.
		\item	\textbf{Nadglednik} - Razumije naputke korisnika i pretvara ih u upute ka modelu. Sadrži razrede koji upravljaju i rukuju korisničkom interakcijom s pogledom i modelom, poput poslovne logike i odgovora na događaje.
	\end{packed_item}

	\eject
		

		

				
		\section{Baza podataka}
			
			\textbf{\textit{dio 1. revizije}}\\
			
		\textit{Potrebno je opisati koju vrstu i implementaciju baze podataka ste odabrali, glavne komponente od kojih se sastoji i slično.}
		
			\subsection{Opis tablica}
			

				\textit{Svaku tablicu je potrebno opisati po zadanom predlošku. Lijevo se nalazi točno ime varijable u bazi podataka, u sredini se nalazi tip podataka, a desno se nalazi opis varijable. Svjetlozelenom bojom označite primarni ključ. Svjetlo plavom označite strani ključ}
				
				
				\begin{longtblr}[
					label=none,
					entry=none
					]{
						width = \textwidth,
						colspec={|X[6,l]|X[6, l]|X[20, l]|}, 
						rowhead = 1,
					} %definicija širine tablice, širine stupaca, poravnanje i broja redaka naslova tablice
					\hline \multicolumn{3}{|c|}{\textbf{korisnik - ime tablice}}	 \\ \hline[3pt]
					\SetCell{LightGreen}IDKorisnik & INT	&  	Lorem ipsum dolor sit amet, consectetur adipiscing elit, sed do eiusmod  	\\ \hline
					korisnickoIme	& VARCHAR &   	\\ \hline 
					email & VARCHAR &   \\ \hline 
					ime & VARCHAR	&  		\\ \hline 
					\SetCell{LightBlue} primjer	& VARCHAR &   	\\ \hline 
				\end{longtblr}
				
				
			
			\subsection{Dijagram baze podataka}
				\textit{ U ovom potpoglavlju potrebno je umetnuti dijagram baze podataka. Primarni i strani ključevi moraju biti označeni, a tablice povezane. Bazu podataka je potrebno normalizirati. Podsjetite se kolegija "Baze podataka".}
			
			\eject
			
			
		\section{Dijagram razreda}
			
			\indent Na naredne tri slike nalaze se dijagrami razreda podijeljeni logički po srodnosti. Neki razredi povezani su i s onima na odvojenim slikama što se da zaključiti po nazivima njihovih metoda. \\
			
			\indent Razredi na slici ??? uokvireni plavom bojom su controlleri. Njihove metode služe za primanje i slanje DTO-ova (\textit{Data Transfer Objects})  prema frontendu u obliku JSON datoteka s html statusnim kodom. Razlikuju se controlleri za korisnike, događaje te postupke prijave i registracije. Sami DTO razredi nalaze se na slici ???, a to su zahtjevi i odgovori za prijavu i registraciju.\\
			
			\indent Razredi na slici ??? su modeli. Njihove metode izravno barataju s bazom podataka i dohvaćaju željene podatke iz nje. Razred \textit{User} predstavlja korisnika aplikacije, \textit{Role} predstavlja različite uloge, \textit{Event} događaje, \textit{EventType} vrste događaja i \textit{Tag} oznake za događaje.
			
			\begin{figure}[h]
				\begin{center}
					\includegraphics[width=\textwidth]{dijagrami/UML dtoovi.png}
					\caption{Dijagram razreda - data transfer objects}
				\end{center}	
			\end{figure}
		
			\begin{figure}[h]
				\begin{center}
					\includegraphics[width=\textwidth]{dijagrami/UML kontroleri, servisi, repozitoriji.png}
					\caption{Dijagram razreda - nadglednici, servisi, repozitoriji}
				\end{center}	
			\end{figure}
		
			\begin{figure}[h]
				\begin{center}
					\includegraphics[width=\textwidth]{dijagrami/UML modeli.png}
					\caption{Dijagram razreda - modeli}
				\end{center}	
			\end{figure}
		
			\eject
			
			\iffalse
			
				\textbf{\textit{dio 2. revizije}}\\			
			
				\textit{Prilikom druge predaje projekta dijagram razreda i opisi moraju odgovarati stvarnom stanju implementacije}
			
			
			
				\eject
		
			\section{Dijagram stanja}
			
			
				\textbf{\textit{dio 2. revizije}}\\
			
				\textit{Potrebno je priložiti dijagram stanja i opisati ga. Dovoljan je jedan dijagram stanja koji prikazuje \textbf{značajan dio funkcionalnosti} sustava. Na primjer, stanja korisničkog sučelja i tijek korištenja neke ključne funkcionalnosti jesu značajan dio sustava, a registracija i prijava nisu. }
			
			
				\eject 
		
			\section{Dijagram aktivnosti}
			
				\textbf{\textit{dio 2. revizije}}\\
			
			 	\textit{Potrebno je priložiti dijagram aktivnosti s pripadajućim opisom. Dijagram aktivnosti treba prikazivati značajan dio sustava.}
			
				\eject
			\section{Dijagram komponenti}
		
				\textbf{\textit{dio 2. revizije}}\\
		
			 	\textit{Potrebno je priložiti dijagram komponenti s pripadajućim opisom. Dijagram komponenti treba prikazivati strukturu cijele aplikacije.}
		\fi
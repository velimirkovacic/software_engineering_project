\chapter{Arhitektura i dizajn sustava}
		
		\textbf{\textit{dio 1. revizije}}\\

		\textit{ Potrebno je opisati stil arhitekture te identificirati: podsustave, preslikavanje na radnu platformu, spremišta podataka, mrežne protokole, globalni upravljački tok i sklopovsko-programske zahtjeve. Po točkama razraditi i popratiti odgovarajućim skicama:}
	\begin{itemize}
		\item 	\textit{izbor arhitekture temeljem principa oblikovanja pokazanih na predavanjima (objasniti zašto ste baš odabrali takvu arhitekturu)}
		\item 	\textit{organizaciju sustava s najviše razine apstrakcije (npr. klijent-poslužitelj, baza podataka, datotečni sustav, grafičko sučelje)}
		\item 	\textit{organizaciju aplikacije (npr. slojevi frontend i backend, MVC arhitektura) }		
	\end{itemize}

	
		

		

				
		\section{Baza podataka}
			
			\textbf{\textit{dio 1. revizije}}\\
			
		\textit{Potrebno je opisati koju vrstu i implementaciju baze podataka ste odabrali, glavne komponente od kojih se sastoji i slično.}
		
		Za potrebe našeg sustava koristit ćemo relacijsku bazu podataka koja svojom strukturom olakšava modeliranje stvarnog svijeta. Gradivna jedinka baze je relacija, odnosno tablica koja je definirana svojim imenom i skupom atributa. Zadaća baze podataka je brza i jednostavna pohrana, izmjena i dohvat podataka za daljnju obradu.Baza podataka ove aplikacije sastoji se od sljedećih entiteta:
		
		\begin{itemize}
			\item Korisnik
			\item Uloga
			\item Vrsta
			\item Oznaka
			\item ImaUlogu
			\item JePrijatelj
			\item JeBlokiranOd
			\item Dogadjaj
			\item Pohadja
			\item ImaOznaku
			
		\end{itemize}
		
		
			\subsection{Opis tablica}
			

				\textit{Svaku tablicu je potrebno opisati po zadanom predlošku. Lijevo se nalazi točno ime varijable u bazi podataka, u sredini se nalazi tip podataka, a desno se nalazi opis varijable. Svjetlozelenom bojom označite primarni ključ. Svjetlo plavom označite strani ključ}
				
				\noindent\textbf{Korisnik} Ovaj entitet sadržava sve važne informacije o korisniku aplikacije. Sadrži atribute: id korisnika, nadimak, korisničko ime, email, salt, lozinka i suspendiran.
				
				\begin{longtblr}[
					label=none,
					entry=none
					]{
						width = \textwidth,
						colspec={|X[6,l]|X[6, l]|X[20, l]|}, 
						rowhead = 1,
					} %definicija širine tablice, širine stupaca, poravnanje i broja redaka naslova tablice
					\hline \multicolumn{3}{|c|}{\textbf{Korisnik}}	 \\ \hline[3pt]
					\SetCell{LightGreen}id korisnik & BIGINT NOT NULL	&  	Lorem ipsum dolor sit amet, consectetur adipiscing elit, sed do eiusmod  	\\ \hline
					nadimak	& VARCHAR(25) NOT NULL &   	\\ \hline
					korisnicko ime & VARCHAR(25) NOT NULL &   \\ \hline  
					email & VARCHAR(255) NOT NULL &   \\ \hline 
					salt & BYTEA NOT NULL	&  		\\ \hline 
					lozinka & BYTEA NOT NULL	&  		\\ \hline 
					suspendiran & BOOLEAN NOT NULL	&  		\\ \hline 
					
				\end{longtblr}
			
			
				\noindent\textbf{Uloga} Ovaj entitet sadržava sve važne informacije o ulogama korisnika. Sadrži atribute: id uloga, naziv uloga i opis uloga.
				
				\begin{longtblr}[
					label=none,
					entry=none
					]{
						width = \textwidth,
						colspec={|X[6,l]|X[6, l]|X[20, l]|}, 
						rowhead = 1,
					} %definicija širine tablice, širine stupaca, poravnanje i broja redaka naslova tablice
					\hline \multicolumn{3}{|c|}{\textbf{Uloga}}	 \\ \hline[3pt]
					\SetCell{LightGreen}id uloga & BIGINT NOT NULL	&  	Lorem ipsum dolor sit amet, consectetur adipiscing elit, sed do eiusmod  	\\ \hline
					naziv uloga	& VARCHAR(255) NOT NULL &   	\\ \hline 
					opis uloga & VARCHAR(255) NOT NULL &   \\ \hline 
					
				\end{longtblr}
			
				\noindent\textbf{Vrsta} Ovaj entitet sadržava sve važne informacije o vrstama događaja. Sadrži atribute: id vrsta, naziv vrsta i opis vrsta.
				
				\begin{longtblr}[
					label=none,
					entry=none
					]{
						width = \textwidth,
						colspec={|X[6,l]|X[6, l]|X[20, l]|}, 
						rowhead = 1,
					} %definicija širine tablice, širine stupaca, poravnanje i broja redaka naslova tablice
					\hline \multicolumn{3}{|c|}{\textbf{Vrsta}}	 \\ \hline[3pt]
					\SetCell{LightGreen}id vrsta & INT NOT NULL	&  	Lorem ipsum dolor sit amet, consectetur adipiscing elit, sed do eiusmod  	\\ \hline
					naziv vrsta	& VARCHAR(255) NOT NULL &   	\\ \hline 
					opis vrsta & VARCHAR(255) NOT NULL &   \\ \hline 
					 
				\end{longtblr}
				
				\noindent\textbf{Oznaka} Ovaj entitet sadržava sve važne informacije o oznakama događaja. Sadrži atribute: id oznaka, naziv oznaka i boja hex.
				
				\begin{longtblr}[
					label=none,
					entry=none
					]{
						width = \textwidth,
						colspec={|X[6,l]|X[6, l]|X[20, l]|}, 
						rowhead = 1,
					} %definicija širine tablice, širine stupaca, poravnanje i broja redaka naslova tablice
					\hline \multicolumn{3}{|c|}{\textbf{Oznaka}}	 \\ \hline[3pt]
					\SetCell{LightGreen}id oznaka & INT NOT NULL	&  	Lorem ipsum dolor sit amet, consectetur adipiscing elit, sed do eiusmod  	\\ \hline
					naziv oznaka	& VARCHAR(255) NOT NULL &   	\\ \hline 
					boja hex & CHAR(7) NOT NULL &   \\ \hline 
					
				\end{longtblr}
				
				
				\noindent\textbf{ImaUlogu} Ovaj entitet sadržava sve važne informacije po kojima saznajemo koji korisnik ima koju ulogu. Sadrži atribute: id korisnik i id uloga.
				
				\begin{longtblr}[
					label=none,
					entry=none
					]{
						width = \textwidth,
						colspec={|X[6,l]|X[6, l]|X[20, l]|}, 
						rowhead = 1,
					} %definicija širine tablice, širine stupaca, poravnanje i broja redaka naslova tablice
					\hline \multicolumn{3}{|c|}{\textbf{ImaUlogu}}	 \\ \hline[3pt]
					\SetCell{LightGreen}id korisnik & BIGINT NOT NULL	&  	Lorem ipsum dolor sit amet, consectetur adipiscing elit, sed do eiusmod  	\\ \hline
					\SetCell{LightGreen}id uloga	& BIGINT NOT NULL &   	\\ \hline 
					 
				\end{longtblr}
				
				\noindent\textbf{JePrijatelj} Ovaj entitet sadržava sve važne informacije o tome koji je korisnik nekom drugom korisniku prijatelj. Sadrži atribute: id korisnik i id prijatelj.
				
				\begin{longtblr}[
					label=none,
					entry=none
					]{
						width = \textwidth,
						colspec={|X[6,l]|X[6, l]|X[20, l]|}, 
						rowhead = 1,
					} %definicija širine tablice, širine stupaca, poravnanje i broja redaka naslova tablice
					\hline \multicolumn{3}{|c|}{\textbf{JePrijatelj}}	 \\ \hline[3pt]
					\SetCell{LightGreen}id korisnik & BIGINT NOT NULL	&  	Lorem ipsum dolor sit amet, consectetur adipiscing elit, sed do eiusmod  	\\ \hline
					\SetCell{LightGreen}id prijatelj	& BIGINT NOT NULL	& 	\\ \hline 
				\end{longtblr}
			
			
				\noindent\textbf{JeBlokiranOd} Ovaj entitet sadržava sve važne informacije o tome koji je korisnik blokiran i od kojeg je korisnika blokiran. Sadrži atribute: id blokiran i id blokiran od.
				
				\begin{longtblr}[
					label=none,
					entry=none
					]{
						width = \textwidth,
						colspec={|X[6,l]|X[6, l]|X[20, l]|}, 
						rowhead = 1,
					} %definicija širine tablice, širine stupaca, poravnanje i broja redaka naslova tablice
					\hline \multicolumn{3}{|c|}{\textbf{JeBlokiranOd}}	 \\ \hline[3pt]
					\SetCell{LightGreen}id blokiran & BIGINT NOT NULL	&  	Lorem ipsum dolor sit amet, consectetur adipiscing elit, sed do eiusmod  	\\ \hline
					\SetCell{LightGreen}id blokiran od	& BIGINT NOT NULL &   	\\ \hline 
				 
				\end{longtblr}
				
				
				\noindent\textbf{Dogadjaj} Ovaj entitet sadržava sve važne informacije o događaju. Sadrži atribute: id događaj, naziv, mjesto, vrijeme početka, vrijeme kraja, opis, promoviran, koordinate, id organizator i id vrsta.
				
				\begin{longtblr}[
					label=none,
					entry=none
					]{
						width = \textwidth,
						colspec={|X[6,l]|X[6, l]|X[20, l]|}, 
						rowhead = 1,
					} %definicija širine tablice, širine stupaca, poravnanje i broja redaka naslova tablice
					\hline \multicolumn{3}{|c|}{\textbf{Dogadjaj}}	 \\ \hline[3pt]
					\SetCell{LightGreen}id dogadjaj & BIGINT NOT NULL	&  	Lorem ipsum dolor sit amet, consectetur adipiscing elit, sed do eiusmod  	\\ \hline
					naziv	& VARCHAR(255) NOT NULL &   	\\ \hline 
					mjesto & VARCHAR(255) NOT NULL &   \\ \hline 
					vrijeme poc & TIMESTAMP NOT NULL	&  		\\ \hline 
					vrijeme kraj & TIMESTAMP NOT NULL	&  		\\ \hline
					opis & VARCHAR(255) NOT NULL &  \\ \hline
					promoviran & BOOLEAN NOT NULL &   \\ \hline 
					koordinate & VARCHAR(255) NOT NULL &  \\ \hline
					\SetCell{LightBlue}id organizator & BIGINT NOT NULL &   \\ \hline 
					\SetCell{LightBlue}id vrsta & INT NOT NULL & 
					\\ \hline
				\end{longtblr}
				
				
				\noindent\textbf{Pohadja} Ovaj entitet sadržava sve važne informacije o tome tko je pohađao koji događaj i kako ga je ocijenio. Sadrži atribute: recenzija, id polaznika i id događaja.
				
				\begin{longtblr}[
					label=none,
					entry=none
					]{
						width = \textwidth,
						colspec={|X[6,l]|X[6, l]|X[20, l]|}, 
						rowhead = 1,
					} %definicija širine tablice, širine stupaca, poravnanje i broja redaka naslova tablice
					\hline \multicolumn{3}{|c|}{\textbf{Pohadja}}	 \\ \hline[3pt]
					recenzija & SMALLINT NOT NULL	&  	Lorem ipsum dolor sit amet, consectetur adipiscing elit, sed do eiusmod  	\\ \hline
				    \SetCell{LightGreen} id pohadjatelja	& BIGINT NOT NULL &   	\\ \hline 
					\SetCell{LightGreen} id dogadjaja & BIGINT NOT NULL &   \\ \hline 
					 
				\end{longtblr}
			
				
				\noindent\textbf{ImaOznaku} Ovaj entitet sadržava sve važne informacije o oznakama određenih događaja. Sadrži atribute: id događaj i id oznaka.
				
				\begin{longtblr}[
					label=none,
					entry=none
					]{
						width = \textwidth,
						colspec={|X[6,l]|X[6, l]|X[20, l]|}, 
						rowhead = 1,
					} %definicija širine tablice, širine stupaca, poravnanje i broja redaka naslova tablice
					\hline \multicolumn{3}{|c|}{\textbf{ImaOznaku}}	 \\ \hline[3pt]
					\SetCell{LightGreen}id dogadjaj & BIGINT NOT NULL	&  	Lorem ipsum dolor sit amet, consectetur adipiscing elit, sed do eiusmod  	\\ \hline
					\SetCell{LightGreen}id oznaka	& INT NOT NULL &   	\\ \hline 
				
				\end{longtblr}
			
				
			
			
				
				
			
			\subsection{Dijagram baze podataka}
				
				
			\begin{figure}[h]
				\includegraphics[width=\textwidth]{dijagrami/Baza podataka/REL shema.png}
				\caption{E-R dijagram baze podataka}
			\end{figure}
				
			\eject
			
			
		\section{Dijagram razreda}
		
			\textit{Potrebno je priložiti dijagram razreda s pripadajućim opisom. Zbog preglednosti je moguće dijagram razlomiti na više njih, ali moraju biti grupirani prema sličnim razinama apstrakcije i srodnim funkcionalnostima.}\\
			
			\textbf{\textit{dio 1. revizije}}\\
			
			\textit{Prilikom prve predaje projekta, potrebno je priložiti potpuno razrađen dijagram razreda vezan uz \textbf{generičku funkcionalnost} sustava. Ostale funkcionalnosti trebaju biti idejno razrađene u dijagramu sa sljedećim komponentama: nazivi razreda, nazivi metoda i vrste pristupa metodama (npr. javni, zaštićeni), nazivi atributa razreda, veze i odnosi između razreda.}\\
			
			
			\begin{figure}[h]
				\includegraphics[width=\textwidth]{dijagrami/UML dtoovi.png}
				\caption{Dijagram razreda - dio Dtoovi}
			\end{figure}
		
		    \begin{figure}[h]
		    	\includegraphics[width=\textwidth]{dijagrami/UML kontroleri, servisi, repozitoriji.png}
		    	\caption{Dijagram razreda - dio Kontroleri, servisi i repozitoriji}
		    \end{figure}
			
			\begin{figure}[h]
				\includegraphics[width=\textwidth]{dijagrami/UML modeli.png}
				\caption{Dijagram razreda - dio Modeli}
			\end{figure}
		
			\iffalse
			
				\textbf{\textit{dio 2. revizije}}\\			
			
				\textit{Prilikom druge predaje projekta dijagram razreda i opisi moraju odgovarati stvarnom stanju implementacije}
			
			
			
				\eject
		
			\section{Dijagram stanja}
			
			
				\textbf{\textit{dio 2. revizije}}\\
			
				\textit{Potrebno je priložiti dijagram stanja i opisati ga. Dovoljan je jedan dijagram stanja koji prikazuje \textbf{značajan dio funkcionalnosti} sustava. Na primjer, stanja korisničkog sučelja i tijek korištenja neke ključne funkcionalnosti jesu značajan dio sustava, a registracija i prijava nisu. }
			
			
				\eject 
		
			\section{Dijagram aktivnosti}
			
				\textbf{\textit{dio 2. revizije}}\\
			
			 	\textit{Potrebno je priložiti dijagram aktivnosti s pripadajućim opisom. Dijagram aktivnosti treba prikazivati značajan dio sustava.}
			
				\eject
			\section{Dijagram komponenti}
		
				\textbf{\textit{dio 2. revizije}}\\
		
			 	\textit{Potrebno je priložiti dijagram komponenti s pripadajućim opisom. Dijagram komponenti treba prikazivati strukturu cijele aplikacije.}
		\fi